\documentclass[a4paper]{article}
\usepackage{pgfplots}
\usepackage{tikz}
\usepackage{import}
\usepackage[font=scriptsize,labelfont=bf]{caption}
\usepackage{array}

\begin{document}


\newcommand*{\figuretitle}[1]{%
    {\centering%   <--------  will only affect the title because of the grouping (by the
    \textbf{#1}%              braces before \centering and behind \medskip). If you remove
    \par\medskip}%            these braces the whole body of a {figure} env will be centered.
}


\title{Libpll per node benchmarks}
\maketitle

Each point on the plots corresponds to one inner node. We first compute the recursive number of children under this node (x axis). Then we run update partials on this node (1000 times) and plot the elapsed time (y axis). We do that for each dataset and with/without sites repeats. 

\begin{figure}[!htb]
\figuretitle{seq 59}
\subimport{../pernode_benchs/}{results59.tex}
\end{figure}
\begin{figure}[!htb]
\figuretitle{seq 128}
\subimport{../pernode_benchs/}{results128.tex}
\end{figure}
\begin{figure}[!htb]
\figuretitle{seq 404}
\subimport{../pernode_benchs/}{results404.tex}
\end{figure}

\end{document}