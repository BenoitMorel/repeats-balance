  \documentclass[a4paper]{article}
\usepackage{pgfplots}
\usepackage{tikz}
\usepackage{import}
\usepackage[font=scriptsize,labelfont=bf]{caption}
\usepackage{array}

\begin{document}


\newcommand*{\figuretitle}[1]{%
    {\centering%   <--------  will only affect the title because of the grouping (by the
    \textbf{#1}%              braces before \centering and behind \medskip). If you remove
    \par\medskip}%            these braces the whole body of a {figure} env will be centered.
}


\title{Libpll sequential benchmarks}
\maketitle


\section{Benchmark description}

The following benchmarks compare several libpll implementations with different modes. They measure the execution time of a full likelihood computation on a fixed tree. To avoid measuring the initialization part, we repeat several times  \texttt{pll\_update\_partials} and \texttt{pll\_compute\_edge\_likelihood} on the same partitions and tree.

\begin{itemize}
\item xflouris means that the implentation used is this one: \\ https://github.com/xflouris/libpll.
\item bmorel means that the implementation  used is this one: \\ https://github.com/BenoitMorel/libpll. It supports sites repeats, and the data structure used is a bit different from xflouris (even without sites repeats): CLVs are not supposed to be sorted by sites, and an additional lookup table is used to access them in most of the core functions.
\item default mode means that the option \texttt{PLL\_ATTRIB\_PATTERN\_TIP} and \texttt{PLL\_ATTRIB\_SITES\_REPEATS} are unset.
\item tip pattern means that the option \texttt{PLL\_ATTRIB\_PATTERN\_TIP} is set.
\item sites repeats means that the option \texttt{PLL\_ATTRIB\_SITES\_REPEATS} is set.
\item M is the size of the buffer allocated to compute the sites repeats class identifiers. When it increases, more nodes can benefit from sites repeats. 
\end{itemize}


\section{Benchmark}

\begin{figure}[!htb]
\figuretitle{CPU architecture, 50 iterations}
\subimport{../sequential_benchs/}{bench_50_iterationscpu.tex}
\end{figure}
\begin{figure}[!htb]
\figuretitle{SSE architecture, 50 iterations}
\subimport{../sequential_benchs/}{bench_50_iterationssse.tex}
\end{figure}
\begin{figure}[!htb]
\figuretitle{AVX architecture, 50 iterations}
\subimport{../sequential_benchs/}{bench_50_iterationsavx.tex}
\end{figure}

\end{document}