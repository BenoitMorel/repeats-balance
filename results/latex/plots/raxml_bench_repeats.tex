 
\documentclass[a4paper]{article}
\usepackage{pgfplots}
\usepackage{tikz}
\usepackage{import}
\usepackage[font=scriptsize,labelfont=bf]{caption}
\usepackage{array}
\usepackage{amsmath}
\usepackage{listings}
\lstset{basicstyle=\ttfamily}
\begin{document}


\newcommand*{\figuretitle}[1]{%
    {\centering%   <--------  will only affect the title because of the grouping (by the
    \textbf{#1}%              braces before \centering and behind \medskip). If you remove
    \par\medskip}%            these braces the whole body of a {figure} env will be centered.
}


\title{Raxml repeats benchmarks}
\maketitle

\noindent We run raxml for each dataset with and without the repeats option, and compare the execution times.\\


\noindent Speedup is calculated as follow :  
$$ speedup = \frac{time_{tipinner}}{time_{repeats}}$$

When we don't indicate times, the cluster cancelled the jobs before the end (after 24h). We then compare the speedup at the last reached step. \newline

Raxml commands used:
\lstset{language=sh}
\begin{lstlisting}
  $  ./raxml-ng-mpi --seed=42 --msa data.phy --model data.part
   --simd AVX --threads 16  --redo --repeats on
  $  ./raxml-ng-mpi --seed=42 --msa data.phy --model data.part
   --simd AVX --threads 16  --redo
\end{lstlisting}


\begin{tabular}{|l|c|c|c|c|c|c|c|}
\hline dataset                    & taxas & sites    & partitions & type & $time_{repeats}$ & $time_{tipinner}$ & speedup\\
\hline 404                        & 404   &  7444    & 11         & DNA  & 766s             &  1131s            & 1.47  \\
\hline 1kite\_science2013         & 144   &  371434  & 50         & DNA  & 11400s           &  20414s           & 1.8  \\
\hline 1kite\_hyme                & 174   &  2248590 & 4116       & DNA  & cancelled        &  cancelled        & 1.53 \\
\hline Antl\_1\_1\_nt             & 40    &  522173  & 658        & DNA  & 1402s            &  2293s            & 1.63  \\
\hline Antl\_1\_1\_aa             & 40    &  762438  & 659        & prot & 10161s           & 14500s            & 1.4 \\
\hline para\_1\_nt                & 193   &  1514275 & 3714       & DNA  & cancelled        &  cancelled        & 1.81  \\
\hline 
\end{tabular}\newline
\newline



%para\_1\_aa ne passe pas en memoire

\end{document}